\section{Eryk Mikołajek}

\subsection{Wyrażenia matematyczne}
\begin{itemize}
    \item[*] Tożsamość Eulera nazywana jest także najpiękniejszym wzorem matematyki: \[e^{i\pi} + 1 = 0\]
    \item[*]  Własność liczby e ze wzoru Stirlinga: \[\lim_{x \to \infty} \frac{(n!)^{\frac{1}{n}}}{n} = \frac{1}{e}\] 
    \item[*] Oszacowanie liczby e Oszacowanie liczby podane przez Castellanosa w 1988: \[e \approx \sqrt[6]{\pi^4 + \pi^5}\]
\end{itemize}

\subsection{Ciekawa zagadka, nad którą warto się zastanowić}
\begin{figure}[h!]
    \centering
    \includegraphics[width=0.3\textwidth]{pictures/zdjecie_zagadka.jpg}
    \caption{Zagadka matematyczna}
    \label{fig:zagadka}
\end{figure}
\subsection{Przydadna tabela na zajęcia z matematyki}
\begingroup
\setlength{\tabcolsep}{15pt}
\renewcommand{\arraystretch}{1.5}
\begin{table}[htbp]
\centering
\begin{tabular}{|l|l|l|l|l|l|}
\hline
\multicolumn{1}{|c|}{\multirow{2}{*}{$\alpha$}} & 0\degree & 30\degree  & 45\degree   & 60\degree  & 90\degree  \\ \cline{2-6} 
\multicolumn{1}{|c|}{}                      & 0 & $\frac{\pi}{6}$  & $\frac{\pi}{4}$   & $\frac{\pi}{3}$  & $\frac{\pi}{2}$  \\ \hline
$\sin\alpha$  & 0 & $\frac{1}{2}$  & $\frac{\sqrt{2}}{2}$  & $\frac{\sqrt{3}}{2}$ & 1  \\ \hline
$\cos\alpha$  & 1 & $\frac{\sqrt{3}}{2}$ & $\frac{\sqrt{2}}{2}$  & $\frac{1}{2}$ & 0  \\ \hline
$\tan\alpha$  & 0 & $\frac{\sqrt{3}}{3}$ & 1  & $\sqrt{3}$ & -  \\ \hline
$\cot\alpha$  & - & $\sqrt{3}$ & 1 & $\frac{\sqrt{3}}{3}$  & 0  \\ \hline
\end{tabular}
\end{table}
\endgroup\label{tab:trygonometria}
\subsection{Kilka akapitów tekstu, w które warto się wczytać}  
\paragraph{} Lorem ipsum dolor sit amet, consectetur adipiscing elit. Nullam quis nulla mi. Nulla consectetur tellus at varius rutrum. Vestibulum egestas diam et dolor mattis, finibus dictum tortor ultricies. \par
Donec pulvinar vitae arcu nec consequat. Cras imperdiet nisi sit amet velit viverra luctus. Quisque in varius sem. Mauris nec nibh dignissim, pretium quam ut, maximus quam. 
\subsection{Podstawowe formatowanie tekstu + lista numerowana}
\begin{enumerate}
    \item \emph{Italicized text}
    \item \textbf{Bold text}
    \item \underline{Underlined text}
\end{enumerate}
\subsection{Odwołania}
Obrazek z zagadką (patrz grafika \ref{fig:zagadka}, strona \pageref{fig:zagadka}) \\
Tabela trygonometryczna (patrz rozdział \ref{tab:trygonometria}, strona \pageref{tab:trygonometria})
