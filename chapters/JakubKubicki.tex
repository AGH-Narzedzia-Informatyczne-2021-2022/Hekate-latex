\section{Jakub Kubicki}

\subsection{Narodziny}
Zespół tworzy rodzeństwo Katarzyna Sienkiewicz i Jacek Sienkiewicz (dzieci Kuby Sienkiewicza), a sama grupa oficjalnie funkcjonuje od marca 2018. Wcześniej rodzeństwo występowało w grupie Hollow Quartet, z którą wydali dobrze przyjęty album Chodź ze mną i wystąpili w telewizyjnym talent show Must Be the Music. Tylko muzyka. Do sierpnia 2018 ich debiutancki utwór „Dziś późno pójdę spać” zanotował dwa miliony wyświetleń w serwisie YouTube, do lutego 2019 uzyskał blisko pięć milionów wyświetleń, a do września 2021 teledysk odtworzono ponad 31 milionów razy. W październiku 2018 ukazał się drugi singel zespołu „Niemożliwe”.

\subsection{Kariera}
W marcu 2019 zespół uplasował się na 3. miejscu w drugiej edycji plebiscytu Sanki 2019 na najciekawsze nowe twarze polskiej sceny muzycznej organizowanym przez „Gazetę Wyborczą”, a w lipcu wystąpił na małej scenie na Pol’and’Rock Festival, gdzie ich koncertu słuchało kilka tysięcy osób. Występ na festiwalu zaowocował 29 listopada 2019 wydaniem albumu Live Pol’and’Rock Festival 2019 z zapisem ich koncertu.

W czerwcu 2020 Kwiat Jabłoni został zwycięzcą plebiscytu Złotego Bączka, co zagwarantowało im udział w internetowej odsłonie Pol’and’Rock Festival na przełomie lipca i sierpnia 2020. W listopadzie wydali singel „Mogło być nic”, który w trakcie jednego tygodnia zdobył ponad milion wyświetleń w serwisie internetowym YouTube. Promował on, wydany 5 lutego 2021, drugi album studyjny o tym samym tytule.

\textbf{Wzór na obliczenie liczby osób na widowni podczas Pol'and 'Rock Festival:}
\[ N = \sum_{n=1}^{\infty} \sqrt{b*n} \]\label{math:wzorek}
\rightline{Gdzie \textit{b} odpowiada liczbie biletów, a \textbf{\textit{N}} liczebnośći widowni. }

\begin{figure}[htbp] 
    \centering
    \includegraphics[width=0.9\textwidth]{pictures/okladka.jpg} % 
    \caption{Okładka najnowszej płyty}
    \label{fig:okladka}
\end{figure}
\newpage
\centering Tabela pokazująca sprzedaż płyt zespołu "Kwiat Jabłoni."
\begin{table}[]
\centering
\begin{tabular}{|l|l|l|}
\hline
\textbf{Rok}  & \textbf{Tytuł}                  & \textbf{sprzedaż} \\ \hline
\textit{2019} & Live Pol’and’Rock Festival 2019 & nieznana          \\ \hline
\textit{2019} & "Niemożliwe"                    & ok. 60 tyś        \\ \hline
\textit{2021} & "Mogło być nic"                 & ok. 15 tyś        \\ \hline
\end{tabular}
\end{table}\label{tab:sprzedaz}

Podsumowanie 1
\begin{itemize}
  \item Kwiat jabłoni jest zespołem niedawno stworzonym
  \item Na swoim koncie mają co najmniej 3 płyty pokazane w tabeli \ref{tab:sprzedaz}
  \item Okładka z najnowszej płyty pokazana jest na stronie \pageref{fig:okladka}
\end{itemize}


Podsumowanie 2
\begin{enumerate}
\item Kwiat jabłoni to zespół muzyczny/
\item ich kariera sięgała szczytu w 2019 roku
\item Liczba osób na widowni podczas Pol'and'Rock jest bliżej nieokreślona pomimo podanego wzoru \ref{math:wzorek}
 \end{enumerate}