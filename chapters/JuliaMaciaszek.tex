\section{Julia Maciaszek}
\label{sec:julmac}
\setlength{\parindent}{4em}
\setlength{\parskip}{1em}
\renewcommand{\labelitemi}{-}

\subsection{Logo}
Równanie okręgu: $(x-a)^2+(y-b)^2=r^2$\\
Oto logo naszego bota (Grafika~\ref{fig:logo}).

\begin{figure}[htbp] % Co oznacza [htbp]? [htbp] kontroluje gdzie jest umieszczane
    \centering
    \includegraphics[width=0.3\textwidth]{pictures/logo.png} % Jak sprawić, żeby obrazek był większy?
    \caption{Nasze logo}
    \label{fig:logo}
\end{figure}

Table~\ref{tab:kolory} Pokazuje kolory użyte do tworzenia loga. % Do czego służy \ref{}?
\begin{table}[htbp]
\centering
\begin{tabular}{c ||c } 
 Fragment loga & Kolor(HEX) \\
 \hline\hline
 Tło & \#010a13 \\ 
 \hline
 Punkty w tle & \#89706e \\
 \hline
 Świetliste linie & \#394a54 \\
 \hline
 Świetliste punkty & \#839a9f \\
 \hline
 Złoto & \#a47b30 \\
\end{tabular}
\label{tab:kol}
\caption{Kolorystyka loga}
\end{table}\label{tab:kolory}

\subsection{Znaczenie}
Można doszukiwać się różnych znaczeń elementów:
\begin{itemize}
  \item Światła w tle jako światła w pokojach siedzących do późna studentów
  \item świetlne linie i punkty jako użytkownicy bota połączeni ze sobą dzięki Hekate
  \item Złoty okrąg z literą H jako klepsydra odmierzająca czas 
\end{itemize}
\newpage

\subsection{Wykonanie}
Do wykonania loga używałam w programie:
\begin{enumerate}
  \item Wypełnienia
  \item Pędzla
  \item Pióra
  \item Rozmycia
  \item Blokady warstwy alpha
\end{enumerate}



\begin{flushleft}
Gdyby bot miał działać na \textbf{większą skalę}, to prawdopodobnie logo zostanie zmienione. Powinno w dobrze ujmować \underline{cele i możliwości} Hekate. Na obecną chwilę jednak posługujemy się takim, jakim jest.\par
Logo powstało, ponieważ \textit{nudziło mi się} i \textit{chciałam porysować}. Ze względu na zajmowanie się Zestawem 1 świetnym pomysłem wydało mi się zrobienie jakiegoś \textbf{\textit{znaku rozpoznawczego}} dla naszego bota (patrz Grafika~\ref{fig:logo}).
\end{flushleft}