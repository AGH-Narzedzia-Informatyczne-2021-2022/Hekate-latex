\section{Olgierd Piofczyk}

\subsection{Wprowadzenie do teorii czcionek}
Nie podoba mi się domyślna czcionka {\LaTeX}'a. Powody:
\begin{itemize}
    \item Jest szeryfowa
    \item Posiada ogromne szeryfy
    \item Ą ma za krótki ogonek
\end{itemize}
5 cudownych czcionek:
\begin{enumerate}
    \item Comic Sans MS
    \item Ubuntu
    \item Helvetica
    \item Comic Sans MS, ale {\color{red} czerwony}
    \item Consolas
\end{enumerate}
Dla każdej czcionki szeryfowej $c$ satysfakcja z patrzenia na nią ($S(c)$) wynosi 0 co można zapisać jako:
{\large
\begin{equation}
    \forall_{c \in C_{sz}}: S(c) = 0
\end{equation}
}
Ponadto:
{\large
\begin{equation}
    \forall_{c\in C_{sz}} \exists_{b \in C_{bsz}}: S(b) \geq S(c)
\end{equation}
}
Oraz:
{\large
\begin{equation}
    S(c) = \infty \leftrightarrow c - \text{Comic Sans MS ({\color{red} czerwony})}
\end{equation}
}
Po więcej matematycznych mądrości zobacz tabelę \ref{tab:mnozenie}
\hrule

\subsection{Jak to jest być skrybą? Dobrze?}
\begin{figure}[h!]
    \centering
    \includegraphics[width=0.3\textwidth]{pictures/skryba.png}
    \caption{Skryba}
    \label{fig:skryba}
\end{figure}
A, wie pan, moim zdaniem to nie ma tak, że \textbf{dobrze}, albo że \textbf{niedobrze}. Gdybym miał powiedzieć, co cenię w życiu najbardziej, powiedziałbym, że \emph{ludzi. Ludzi}, którzy podali mi pomocną dłoń, kiedy sobie nie radziłem, kiedy byłem sam, i co ciekawe, \underline{to właśnie przypadkowe spotkania} wpływają na nasze życie. Chodzi o to, że kiedy wyznaje się pewne wartości, nawet pozornie uniwersalne, bywa, że nie znajduje się zrozumienia, które by tak rzec, które pomaga się nam rozwijać. Ja miałem szczęście, by tak rzec, ponieważ je znalazłem, i \textbf{dziękuję życiu!} Dziękuję mu; życie to śpiew, życie to taniec, życie to miłość! 

Wielu ludzi pyta mnie o to samo: \textit{ale jak ty to robisz, skąd czerpiesz tę radość?} A ja odpowiadam, \textbf{że to proste}! To \underline{umiłowanie życia}. To właśnie ono sprawia, że dzisiaj na przykład buduję maszyny, a jutro – kto wie? Dlaczego by nie – oddam się pracy społecznej i będę, ot, choćby, sadzić... doć— m-marchew...

\begin{flushright}
    --- Otis (Patrz: Grafika \ref{fig:skryba})
\end{flushright}

\subsection{Mnożenie}
\begin{table}[h!]
\centering
\caption{Tabliczka mnożenia}
\label{tab:mnozenie}
\begin{tabular}{|l|l|l|l|l|l|l|l|l|l|} 
\hline
*          & \textbf{1} & \textbf{2} & \textbf{3} & \textbf{4} & \textbf{5} & \textbf{6} & \textbf{7} & \textbf{8} & \textbf{9}  \\ 
\hline
\textbf{1} & 1          & 2          & 3          & 4          & 5          & 6          & 7          & 8          & 9           \\ 
\hline
\textbf{2} & 2          & 4          & 6          & 8          & 10         & 12         & 14         & 16         & 18          \\ 
\hline
\textbf{3} & 3          & 6          & 9          & 12         & 15         & 18         & 21         & 24         & 27          \\ 
\hline
\textbf{4} & 4          & 8          & 12         & 16         & 20         & 24         & 28         & 32         & 36          \\ 
\hline
\textbf{5} & 5          & 10         & 15         & 20         & 25         & 30         & 35         & 40         & 45          \\ 
\hline
\textbf{6} & 6          & 12         & 18         & 24         & 30         & 36         & 42         & 48         & 54          \\ 
\hline
\textbf{7} & 7          & 14         & 21         & 28         & 35         & 42         & 49         & 56         & 63          \\ 
\hline
\textbf{8} & 8          & 16         & 24         & 32         & 40         & 48         & 56         & 64         & 72          \\ 
\hline
\textbf{9} & 9          & 18         & 27         & 36         & 45         & 54         & 63         & 72         & 81          \\
\hline
\end{tabular}
\end{table}